\documentclass[article]{pastex}
\author{Pascal Molin}
\title{Make a gp distribution for osX}
\begin{document}

\section{Description}

The distribution consists in

a `pari-x.x.x.dmg` file containing all installation material


a `Pari.app` bundle to be installed, containing gp binaries and related
libraries.

%\setprogfile[make]{Makefile}
\begin{code}[make]
VERSION=2.5.3
\end{code}

\section{The pari bundle}

Following \url{}
\begin{alltt}
PariGP.app
   PariGP
   PariGP.png
   MySearchIcon.png
   Info.plist
   Default.png
   MainWindow.nib
   Settings.bundle
   MySettingsIcon.png
   iTunesArtwork
   en.lproj
     MyImage.png
   fr.lproj
     MyImage.png
\end{alltt}


\begin{alltt}
PariGP.app/
  Contents/
    Info.plist
    MacOS/
      PariGP
    Resources/
      bin/
      lib/
      foo.icns
\end{alltt}

\begin{code}[sh]
mkdir -p /Applications/PariGP.app/Contents/MacOS
mkdir -p /Applications/PariGP.app/Contents/Resources
\end{code}

\begin{code}[make]
APP:=/Applications/PariGP.app
PariGPdir:=$(APP)/Contents
MacOSdir:=$(PariGPdir)/MacOS
Resourcesdir:=$(PariGPdir)/Resources
Resourceslibdir:=$(Resourcesdir)/lib
GPexe:=$(Resourcesdir)/bin/gp
GPrun:=$(MacOSdir)/PariGP

$(MacOSdir):
	mkdir -p $@

$(Resourcesdir):
	mkdir -p $@

$(Resourceslibdir):
	mkdir -p $@
  
\end{code}




The Info.plist file is
\begin{code}[xml]
<?xml version="1.0" encoding="UTF-8"?>
<!DOCTYPE plist PUBLIC "-//Apple Computer//DTD PLIST 1.0//EN" 
                       "http://www.apple.com/DTDs/PropertyList-1.0.dtd">
<plist version="1.0">
<dict>
    <key>CFBundleExecutable</key>
    <string>PariGP</string>
    <key>CFBundleIdentifier</key>
    <string>math.PariGP</string>
    <key>CFBundleName</key>
    <string>PariGP</string>
    <key>CFBundleIconFile</key>
    <string>PariGP.icns</string>
    <key>CFBundleShortVersionString</key>
    <string>2.5.3</string>
    <key>CFBundleSignature</key>
    <string>pari</string>
    <key>CFBundleVersion</key>
    <string>2.5.3</string>
    <key>CFBundleInfoDictionaryVersion</key>
    <string>6.0</string>
    <key>CFBundlePackageType</key>
    <string>APPL</string>
  </dict>
</plist>
\end{code}

\begin{code}[sh]
mkdir -p /Applications/PariGP.app/Contents
sed 's/VERSION/2.5.3/' files/Info.plist > /Applications/PariGP.app/Contents/Info.plist
\end{code}

\begin{code}[make]
INFO=$(PariGPdir)/Info.plist

$(INFO): files/Info.plist ${Resourcesdir}
	sed 's/VERSION/$(VERSION)/' $< > $@

\end{code}


\section{Compiling gp with libraries}

Compilation using Apple's LLVM gcc is broken, I need
to compile gp using gnu gcc.

The second important issue is readline support.

For the compilation, I install both gcc and readline
using brew
\begin{code}[sh]
brew install gcc
brew install readline
\end{code}

Now the readline libraries are installed in
\begin{code}[sh]
/usr/local/Cellar/readline/6.2.4/lib/libreadline.6.2.dylib
\end{code}
and a decent gcc can be found at
\begin{code}[sh]
/usr/local/bin/gcc-4.7
\end{code}

\subsection{Compilation}

I do this in a subdirectory

Get the sources
\begin{code}[sh] 
git clone http://pari.math.u-bordeaux.fr/git/pari.git sources
\end{code}

Select the version
\begin{code}[sh] 
cd sources
git tag
git co -b pari-2.5.3
\end{code}

configure the compilation. I need to install to the final folder
/Applications/PariGP.app for help.
\begin{code}[sh]
export READLINE=--with-readline=/usr/local/Cellar/readline/6.2.4
export CC=/usr/local/bin/gcc-4.7
export PREFIX=--prefix=/Applications/PariGP.app/Contents/Resources
#export PREFIX=--prefix=../PariGP.app/Contents/Resources
#export SHAREPREFIX=--share-prefix=../PariGP.app/Contents/Resources
./Configure $READLINE $PREFIX
\end{code}

or
\begin{code}[sh]
export CC=/usr/local/bin/gcc-4.7
./Configure --prefix=/Applications/PariGP.app/Contents/ --with-readline-lib=/Applications/PariGP.app/Contents/Resources/lib --with-gmp-lib=/Applications/PariGP.app/Contents/Resources/lib  
\end{code}

and compile and install into the directory MacOS
\begin{code}[sh]
make clean
make -j6 gp
make all
make test-all
make install
\end{code}

\begin{code}[make]
BUILD=sources
READLINE=--with-readline=/usr/local/Cellar/readline/6.2.4
CC=/usr/local/bin/gcc-4.7
PREFIX=--prefix=$(Resourcesdir)


sources/build-%:
	cd sources && git co pari-$*
	export CC=$(CC)
	cd sources && ./Configure ${READLINE} ${PREFIX} --builddir=build-$*
	cd sources && make clean
	cd sources && make -j6 gp
	cd sources && make all
	cd sources && make dobench

${GPinstall}: sources/build-${VERSION}
	export CC=$(CC)
	cd sources && make install

\end{code}

\subsection{Make bundle}

The binaries are now in the directory `Resources/`.
Now we need to include libraries and fix their paths.

\begin{code}[sh]
cd /Applications/PariGP.app/Contents/Resources 
\end{code}

The library needed are then
\begin{code}[sh]
otool -L ./bin/gp
\end{code}

\begin{alltt}
 libpari-gmp-2.6.dylib
 libgcc_s.1.dylib
 libreadline.6.2.dylib
 libX11.6.dylib
 libSystem.B.dylib
\end{alltt}

The last two are present in the system, the other will be put in the
`Resources/lib/` directory
\begin{code}[sh]
cp -f /usr/local/opt/readline/lib/libreadline.6.2.dylib lib/
cp -f /usr/local/Cellar/gcc/4.7.1/gcc/lib/libgcc_s.1.dylib lib/
cp -f /usr/local/lib/libgmp.10.dylib lib/
\end{code}

\begin{code}[make]
ReadLineLib=libreadline.6.2.dylib
RLorig=/usr/local/opt/readline/lib/$(ReadLineLib)
RLdest=$(Resourcesdir)/lib/$(ReadLineLib)
GccLib=libgcc_s.1.dylib
Gccorig=/usr/local/Cellar/gcc/4.7.1/gcc/lib/$(GccLib)
Gccdest=$(Resourcesdir)/lib/$(GccLib)

$(RLdest): $(RLorig) ${Resourceslibdir}
	cp -f $< $@

$(Gccdest): $(Gccorig) ${Resourceslibdir}
	cp -f $< $@
  
\end{code}

% no need, in fact
%We need to notify the path change
%\begin{code}[sh]
%install_name_tool \
%  -id @executable_path/../lib/libreadline.6.2.dylib \
%  lib/libreadline.6.2.dylib
%install_name_tool \
%  -id @executable_path/../lib/libgcc_s.1.dylib \
%  lib/libgcc_s.1.dylib
%\end{code}

and the gp binary has to be modified accordingly
\begin{code}[sh]
# for readline
#install_name_tool -change /usr/local/opt/readline/lib/libreadline.6.2.dylib @executable_path/../lib/libreadline.6.2.dylib bin/gp
install_name_tool -change /usr/local/opt/readline/lib/libreadline.6.2.dylib /Applications/PariGP.app/Contents/Resources/lib/libreadline.6.2.dylib bin/gp
# gcc
install_name_tool -change /usr/local/Cellar/gcc/4.7.1/gcc/lib/libgcc_s.1.dylib /Applications/PariGP.app/Contents/Resources/lib/libgcc_s.1.dylib bin/gp
# now on the library
install_name_tool -change /usr/local/Cellar/gcc/4.7.1/gcc/lib/libgcc_s.1.dylib /Applications/PariGP.app/Contents/Resources/lib/libgcc_s.1.dylib lib/libpari-gmp.dylib
install_name_tool -change /usr/local/lib/libgmp.10.dylib /Applications/PariGP.app/Contents/Resources/lib/libgmp.10.dylib lib/libpari-gmp.dylib
\end{code}

Check
\begin{code}[sh]
otool -L bin/gp
otool -L lib/libpari-gmp.dylib
\end{code}

\begin{code}[make]
$(GPexe): $(RLdest) $(Gccdest) PariGP-$(VERSION) ${Resourcesdir}
	cd $(Resourcesdir) && install_name_tool -change $(RLorig) $(RLdest) bin/gp
	cd $(Resourcesdir) && install_name_tool -change $(Gccorig) $(Gccdest) bin/gp

\end{code}


\subsection{Launcher script}

We now create a tiny script that points directs to gp from the bundle
in `PariGP.app/Contents/MacOS/PariGP`

Using sh
\begin{code}[sh] 
#!/bin/sh
ROOT="$(cd "$(dirname "$0")" 2>/dev/null && pwd)"
open -a Terminal ${ROOT}/../Resources/bin/gp
\end{code}

\begin{code}[make]
$(GPrun): files/GPrun.sh ${GPexe} ${MacOSdir}
	  cp -f $< $@
  
\end{code}


Using applescript
\begin{code}[sh]
tell application "Terminal"
  #do script quoted form of POSIX path of (path to me) & "Contents/Resources/bin/gp '" & (name of me) & "'; sleep 1; exit"
  do script "ls " & quoted form of POSIX path of (path to me) & "; PariGP.app/Contents/Resources/bin/gp; sleep 5; exit"
end tell
\end{code}

we compile it to the PariGP launcher



With shell, this is not nice

\begin{code}[sh]
#!/bin/sh
ROOT="$(cd "$(dirname "$0")" 2>/dev/null && pwd)"
open -a Terminal ${ROOT}/../Resources/bin/gp
\end{code}

Copy the script and chmod it to be executable
\begin{code}[sh]
mkdir -p /Applications/PariGP.app/Contents/MacOS
cp files/GPrun.sh /Applications/PariGP.app/Contents/MacOS/PariGP 
chmod +x /Applications/PariGP.app/Contents/MacOS/PariGP
\end{code}


Now gp runs correctly, the bundle is OK.

\section{Optional stuff}

\subsection{Icons}

Create a directory containing the icon in png with many different resolutions
(there should be 16, 32, 128, 256)

\begin{alltt}
icon_16x16.png
icon_16x16@2x.png
icon_32x32.png
icon_32x32@2x.png
icon_128x128.png
icon_128x128@2x.png
icon_256x256.png
icon_256x256@2x.png
icon_512x512.png
icon_512x512@2x.png
\end{alltt}

I have an Inkscape svg file containing a PariGPicon square object.
The export is done by
\begin{code}[sh]
for i in {16,32,64,128,256,512,1024}; \
  /Applications/Inkscape.app/Contents/Resources/bin/inkscape \
  --without-gui \
  --export-id=PariGPicon \
  --export-png=icon_$i.png \
  --export-width=$i \
  icon.svg;
\end{code}

\begin{code}[sh]
for i in {16,32,128,256,512}; \
cp icon_${i}.png PariGP.iconset/icon_${i}x${i}.png && \
cp icon_$((2*i)).png PariGP.iconset/icon_${i}x${i}@2x.png;
\end{code}


\begin{code}[sh]
iconutil -c icns -o PariGP.icns PariGP.iconset
\end{code}


\begin{code}[sh]
cp images/PariGP.icns /Applications/PariGP.app/Contents/Resources/
\end{code}

\begin{code}[make]
ICONS:=$(Resourcesdir)/PariGP.icns

$(ICONS): images/PariGP.icns ${Resourcesdir}
	cp -f $< $@

\end{code}

\begin{code}[make]
# The app directory
$(APP): $(ICONS) $(INFO) $(GPrun)
  
\end{code}



\subsection{Install background}

Convert the file `install.svg` to background file `install.png`
of width 600 pixels.

\begin{code}[sh]
/Applications/Inkscape.app/Contents/Resources/bin/inkscape \
  --without-gui \
  --export-id=install \
  --export-png=install800.png \
  --export-width=800 \
  install.svg
\end{code}

It has to be 72dpi. Resize to `400x233` using gimp.
\begin{code}[sh]
convert -units PixelsPerInch install600x350.png -resample 72 install.png
\end{code}

\section{Put it in a dmg image}

Lets go back in the main directory `gpOSX`.

\subsection{Create writeable dmg}

We get the size of the bundle
\begin{code}[sh]
du -hs PariGP.app
\end{code}

and create a larger disk image containing the PariGP bundle.
\begin{code}[sh]
SIZE=20m
hdiutil create -srcfolder "/Applications/PariGP.app" -volname "PariGP" -fs HFS+ \
  -fsargs "-c c=64,a=16,e=16" -format UDRW -size ${SIZE} pack.temp.dmg
\end{code}


We mount the disk image and store its name under `device`
\begin{code}[sh]
device=$(hdiutil attach -readwrite -noverify -noautoopen "pack.temp.dmg" | \
         egrep '^/dev/' | sed 1q | awk '{print $1}')
\end{code}

It is mounted on `/Volumes/PariGP`.

\begin{code}[sh]
dmgdir=/Volumes/PariGP
mkdir -p ${dmgdir}/.background
cp images/install.png ${dmgdir}/.background/install.png
\end{code}

We give the following set of instructions to osascript
\begin{code}[sh]
tell application "Finder"
     tell disk "PariGP"
           open
           set current view of container window to icon view
           set toolbar visible of container window to false
           set statusbar visible of container window to false
           set the bounds of container window to {400, 100, 800, 333}
           set theViewOptions to the icon view options of container window
           set arrangement of theViewOptions to not arranged
           set icon size of theViewOptions to 72
           set background picture of theViewOptions to file ".background:install.png"
           make new alias file at container window to POSIX file "/Applications" with properties {name:"Applications"}
           set position of item "PariGP.app" of container window to {80, 110}
           set position of item "Applications" of container window to {320, 110}
           update without registering applications
           delay 5
           eject
     end tell
   end tell
\end{code}

and run it
\begin{code}[sh]
osascript files/installer.applescript
\end{code}

Now we set permissions, compress and release the final dmg

\begin{code}[sh]
chmod -Rf go-w /Volumes/PariGP
sync
sync
hdiutil detach ${device}
hdiutil convert "pack.temp.dmg" -format UDZO -imagekey zlib-level=9 -o PariGP-2.5.3
rm -f pack.temp.dmg 
\end{code}

\begin{code}[make]
APPSIZE=20m
VOLUME=/Volumes/PariGP
tmp-$(VERSION).dmg: ${APP}
	hdiutil create -srcfolder "$<" -volname "PariGP" -fs HFS+ \
	-fsargs "-c c=64,a=16,e=16" -format UDRW -size ${APPSIZE} $@
	sleep 5
	# mount
	device=$$(hdiutil attach -readwrite -noverify -noautoopen "$<" | \
	         egrep '^/dev/' | sed 1q | awk '{print $$1}')
	sleep 5
	mkdir -p ${VOLUME}/.background
	cp images/install.png ${VOLUME}/.background/install.png
	osascript files/installer.applescript
	-chmod -Rf go-w ${VOLUME}
	sync
	sync
	hdiutil detach $${device}
  
\end{code}

\begin{code}[make]

PariGP-%.dmg: tmp-%.dmg
	hdiutil convert "$<" -format UDZO -imagekey zlib-level=9 -o PariGP-%


all: PariGP-${VERSION}.dmg

\end{code}


\section{Cleanup}

\begin{code}[make]
clean:
	rm -rf /Applications/PariGP.app
  
\end{code}

\end{document}
